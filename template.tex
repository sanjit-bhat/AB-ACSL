% Inspired by Dan Spielman's template

\documentclass[10pt]{article}
\usepackage[T1]{fontenc}
\usepackage{amssymb}
\usepackage{amsmath}
\usepackage{graphicx}

\usepackage{tikz}
\usetikzlibrary{arrows}
\usepackage{subfigure}
\usepackage{stackrel}
\usepackage{blindtext}

\oddsidemargin=0.15in
\evensidemargin=0.15in
\topmargin=-.5in
\textheight=9in
\textwidth=6.25in

\usepackage[colorlinks=true,breaklinks,pdfpagemode=none,linkcolor=blue, urlcolor = blue, citecolor=blue]{hyperref}
\usepackage{enumerate}

%\usepackage{enumitem}
%\setlist{itemsep=0mm}

%\usepackage[usenames,dvipsnames]{pstricks}
%\usepackage{epsfig}
\usepackage{amsmath,amsfonts,amssymb,bm}
%\usepackage{pst-grad} % For gradients
%\usepackage{pst-plot} % For axes

% Enviroment definitions (add your own here)

\newtheorem{theorem}{Theorem}
\newtheorem{corollary}[theorem]{Corollary}
\newtheorem{lemma}[theorem]{Lemma}
\newtheorem{observation}[theorem]{Observation}
\newtheorem{proposition}[theorem]{Proposition}
\newtheorem{definition}[theorem]{Definition}
\newtheorem{claim}[theorem]{Claim}
\newtheorem{fact}[theorem]{Fact}

\newenvironment{proof}{\noindent{\bf Proof}\hspace*{1em}}{\qed\bigskip}

% New commands (add your own here)

\newcommand{\eps}{\varepsilon}
\newcommand{\bbR}{\mathbb{R}}
\newcommand{\hv}{\hat{v}}
\newcommand{\hL}{\hat{L}}
\newcommand{\hlambda}{\hat{\lambda}}
\newcommand{\homega}{\hat{\omega}}
\newcommand{\hp}{\hat{p}}
\newcommand{\hW}{\hat{W}}
\newcommand{\cK}{\mathcal{K}}
\newcommand{\qed}{\rule{7pt}{7pt}}
\newcommand{\cF}{\mathcal{F}}

\begin{document}

    \noindent
    \begin{center}

        \hrulefill

        \vspace{5pt}

        \makebox[\textwidth]{ {\bf AB IdeaLab, Competitive Programming Team, Fall 18--Spring 19} \hfill  January 18, 2019}
        \vspace{0pt}

        {\Large \hfill  Lecture 3: Lisp\hfill}
        \vspace{10pt}

        {\large \hfill  American Computer Science League, January Contest\hfill}
        \vspace{10pt}

        \makebox[\textwidth]{ {\it Lecturer: Sanjit Bhat \hfill Editor: Alexander Sun} } % primary editor is the 'lecturer'

        \vspace{-3pt}
        \hrulefill
    \end{center}

\section{Exercises}
\begin{enumerate}
    \item
\end{enumerate}

\newpage
\section{Solutions}
\begin{enumerate}
    \item
\end{enumerate}

%\begin{thebibliography}{9}
%    \bibitem{deep_learning}
%    Lecun, Y., Bengio, Y., and Hinton, G. (2015).
%    Deep learning.
%    Nature, 521(7553), 436-444.
%\end{thebibliography}
\end{document}