% Inspired by Dan Spielman's template

\documentclass[10pt]{article}
\usepackage[T1]{fontenc}
\usepackage{amssymb}
\usepackage{amsmath}
\usepackage{graphicx}

\usepackage{tikz}
\usetikzlibrary{arrows}
\usepackage{subfigure}
\usepackage{stackrel}
\usepackage{blindtext}

\oddsidemargin=0.15in
\evensidemargin=0.15in
\topmargin=-.5in
\textheight=9in
\textwidth=6.25in

\usepackage[colorlinks=true,breaklinks,pdfpagemode=none,linkcolor=blue, urlcolor = blue, citecolor=blue]{hyperref}
\usepackage{enumerate}

%\usepackage{enumitem}
%\setlist{itemsep=0mm}

%\usepackage[usenames,dvipsnames]{pstricks}
%\usepackage{epsfig}
\usepackage{amsmath,amsfonts,amssymb,bm}
%\usepackage{pst-grad} % For gradients
%\usepackage{pst-plot} % For axes

% Enviroment definitions (add your own here)

\newtheorem{theorem}{Theorem}
\newtheorem{corollary}[theorem]{Corollary}
\newtheorem{lemma}[theorem]{Lemma}
\newtheorem{observation}[theorem]{Observation}
\newtheorem{proposition}[theorem]{Proposition}
\newtheorem{definition}[theorem]{Definition}
\newtheorem{claim}[theorem]{Claim}
\newtheorem{fact}[theorem]{Fact}

\newenvironment{proof}{\noindent{\bf Proof}\hspace*{1em}}{\qed\bigskip}

% New commands (add your own here)

\newcommand{\eps}{\varepsilon}
\newcommand{\bbR}{\mathbb{R}}
\newcommand{\hv}{\hat{v}}
\newcommand{\hL}{\hat{L}}
\newcommand{\hlambda}{\hat{\lambda}}
\newcommand{\homega}{\hat{\omega}}
\newcommand{\hp}{\hat{p}}
\newcommand{\hW}{\hat{W}}
\newcommand{\cK}{\mathcal{K}}
\newcommand{\qed}{\rule{7pt}{7pt}}
\newcommand{\cF}{\mathcal{F}}

\begin{document}

    \noindent
    \begin{center}

        \hrulefill

        \vspace{5pt}

        \makebox[\textwidth]{ {\bf AB IdeaLab, Competitive Programming Team, Fall 18--Spring 19} \hfill  January 18, 2019}
        \vspace{0pt}

        {\Large \hfill  Lecture 2: Bit-String Flicking\hfill}
        \vspace{10pt}

        {\large \hfill  American Computer Science League, January Contest\hfill}
        \vspace{10pt}

        \makebox[\textwidth]{ {\it Lecturer: Alexander Sun \hfill Editor: Sanjit Bhat} } % primary editor is the 'lecturer'

        \vspace{-3pt}
        \hrulefill
    \end{center}


\section{Introduction/Lecture}
Bit String Flicking is a general term to denote operations that can be done to bit strings/ binary strings. They can involve multiple strings or just a single string. There are 3 general types of operations, your basic truth operators, shifts, and circulates.
\subsection{Truth Operations}
Bit String flicking only involves 4 main operations(they should seem familiar), AND, OR, NOT, EXOR, and all the inverses of them. In contest AND $\rightarrow$ $\&$, OR $\rightarrow$ $\mid$, NOT $\rightarrow$ $\sim$,  EXOR $\rightarrow$ $\oplus$.

E.X. 101 | 010 = 111
\subsection{Shift}
Shifting is very straight forward involving 2 operations. You can shift right or left. A shift, literally shifts the digits the specified amount in the specified direction, while maintaining the same amount of digits, filling in 0's fordigits shifted out of the number.
\\
\\
E.X. LSHIFT-3 100010 =  010000, RSHIFT-3 100010 = 000100
\\
\\
The trick to solving these problems, is to cover up the amount of digits specified on the right or left side, then just fill in 0's for the rest of the digits.
\subsection{Circulate}
Circulating is extremely similar to shifting, except no digits will be deleted by getting shifted out of the number. Instead they are added back (or circulated) on to the other end of the number, either in the right or left direction, a specified amount.
\\
\\
E.X. RCIRC-3    10111 = 11110         LCIRC-3   10111 = 11101
\\
\\
The method to solving these problems, are to take the number of digits on the specified side and shifting them directly behind the remaining digits on the other side.
\subsection{Miscellaneous/General Tips}
Order of Operations: NOT, SHIFT / CIRC, AND, XOR, OR
\\
Solving problems with variables often have more then one solution. Some questions will ask you to list out all values, and others will ask for the amount of possible values. Generally just work from the outside in, and if necessary assign values abced. for each digit the expression is equivalent to, and use that to help you.
\section{Exercises}
Some borrowed from \href{http://minich.com/education/wyo/acsl/bitstringflicking/bitstringwksht1.htm}{here}.
\begin{enumerate}
\item 10010 OR 11101
\item 10001 AND 11011 OR 11001
\item 11100 OR 10101 AND 10111
\item NOT 101 AND 100
\item 101011 XOR 110101
\item 101011 OR 100111 XOR 100010
\item NOT (11101 XNOR 10100)
\item 1011 OR 1101 XOR 1000 AND 1010 OR 1110 AND NOT 1000
\item Find all possible values of x: (RCIRC-2(LSHIFT-1 (NOT X)))=00101
\item List all possible unique values of x that solve the following equation: \\
(LSHIFT-1 (10110 XOR (RCIRC-3 x) AND 11011)) = 01100
\item (RSHIFT-1 (LCIRC-4 (RCIRC-2 01101)))
\item ((RCIRC-14 (LCIRC-23 01101)) $\mid$ (LSHIFT-1 10011) $\&$ (RSHIFT-2 10111)) \\
Hint: Order of operations, generally parenthesis dictate order
\end{enumerate}

%\begin{thebibliography}{9}
%    \bibitem{deep_learning}
%    Lecun, Y., Bengio, Y., and Hinton, G. (2015).
%    Deep learning.
%    Nature, 521(7553), 436-444.
%\end{thebibliography}
\end{document}
