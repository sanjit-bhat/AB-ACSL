% Inspired by Dan Spielman's template

\documentclass[10pt]{article}
\usepackage[T1]{fontenc}
\usepackage{amssymb}
\usepackage{amsmath}
\usepackage{graphicx}

\usepackage{tikz}
\usetikzlibrary{arrows}
\usepackage{subfigure}
\usepackage{stackrel}
\usepackage{blindtext}

\oddsidemargin=0.15in
\evensidemargin=0.15in
\topmargin=-.5in
\textheight=9in
\textwidth=6.25in

\usepackage[colorlinks=true,breaklinks,pdfpagemode=none,linkcolor=blue, urlcolor = blue, citecolor=blue]{hyperref}
\usepackage{enumerate}

%\usepackage{enumitem}
%\setlist{itemsep=0mm}

%\usepackage[usenames,dvipsnames]{pstricks}
%\usepackage{epsfig}
\usepackage{amsmath,amsfonts,amssymb,bm}
%\usepackage{pst-grad} % For gradients
%\usepackage{pst-plot} % For axes

% Enviroment definitions (add your own here)

\newtheorem{theorem}{Theorem}
\newtheorem{corollary}[theorem]{Corollary}
\newtheorem{lemma}[theorem]{Lemma}
\newtheorem{observation}[theorem]{Observation}
\newtheorem{proposition}[theorem]{Proposition}
\newtheorem{definition}[theorem]{Definition}
\newtheorem{claim}[theorem]{Claim}
\newtheorem{fact}[theorem]{Fact}

\newenvironment{proof}{\noindent{\bf Proof}\hspace*{1em}}{\qed\bigskip}

% New commands (add your own here)

\newcommand{\eps}{\varepsilon}
\newcommand{\bbR}{\mathbb{R}}
\newcommand{\hv}{\hat{v}}
\newcommand{\hL}{\hat{L}}
\newcommand{\hlambda}{\hat{\lambda}}
\newcommand{\homega}{\hat{\omega}}
\newcommand{\hp}{\hat{p}}
\newcommand{\hW}{\hat{W}}
\newcommand{\cK}{\mathcal{K}}
\newcommand{\qed}{\rule{7pt}{7pt}}
\newcommand{\cF}{\mathcal{F}}

\begin{document}

    \noindent
    \begin{center}

        \hrulefill

        \vspace{5pt}

        \makebox[\textwidth]{ {\bf AB IdeaLab, Competitive Programming Team, Fall 18--Spring 19} \hfill  January 18, 2019}
        \vspace{0pt}

        {\Large \hfill  Lecture 1: Prefix/Postfix/Infix Notation\hfill}
        \vspace{10pt}

        {\large \hfill  American Computer Science League, January Contest\hfill}
        \vspace{10pt}

        \makebox[\textwidth]{ {\it Lecturer: Alexander Sun \hfill Editor: Sanjit Bhat} } % primary editor is the 'lecturer'

        \vspace{-3pt}
        \hrulefill
    \end{center}

\section{Introduction:}
This problem section refers to the notation/format of the operations in an equation.
\href{http://interactivepython.org/runestone/static/pythonds/BasicDS/InfixPrefixandPostfixExpressions.html}{This source} gives good explanation, and real uses of this seemingly toxic topic.

\subsection{Infix:}
Infix refers to notation of writing the operators lying in between the operands. This is the normal notation that we use. E.X: A + B

\subsection{Prefix:}
In prefix notation the operator lies in front of the operands. E.X: + A B

\subsection{Postfix:}
In postfix notation the operator lies behind the operands. E.X: A B +

\section{Evaluation:}
Evaluation of different notations can be quite confusing. One method to simplify the operation is to insert parenthesis around each operation. Remember that each operation only applies to 2 numbers. Solve the problem by taking apart the problem in layers and inserting parenthesis where each operation falls in the later.

\subsection{Identification:}
The first step is to determine whether the equation is in prefix or postfix. One easy way to tell is if there are operators leading or trailing the first and last numbers in the equation. If there are number leading then it is prefix, and vice-versa for postfix.

\subsection{Prefix:}
-*DA/+BCD = -(*DA)/(+BC)D = -((*DA)(/(+BC)D)) = (D*A)-((B+C)/D) Multiply A and D and subtract the quotient of the sum of B and C and D.

\subsection{Postfix:}
AB*CD/+ = ((AB *) (CD /) +) = (A*B)+(C/D)
Multiply A and B, Divide C by D, then add the results

\section{Exercises:}
Not too many challenging problems in this topic can be written. It's pretty straightforward

\begin{enumerate}

\item Evaluate: 9 18 6 27 3 / * + / 24 2 12 6 / + / /

\item Translate into infix: * + A D - + B C E

\item Find all integer values of Y for which the following prefix expression has
a value of zero\\
* + Y 4 - 6 Y

\item Given A=4, B=14 and C=2, evaluate the following prefix expression: \\
* / - + A B C * A C B

\item Translate the following infix expression into postfix:\\
\begin{equation*}
\frac{(A-\frac{B}{C}+D)^{\frac{1}{2}}}{A+B}
\end{equation*}

\item Evaluate the following prefix expression, when A=10, B=2, C=12 and D=2:\\
\begin{equation*}
+ / \uparrow - A B 2 \uparrow / - C D / A B 3 / + A C B
\end{equation*}
\end{enumerate}

\subsection{Convert to Prefix and Postfix}
\label{sec:convert}

\begin{enumerate}
\item Infix Expression: ( AX + ( B * C ) )

\item Infix Expression: ( ( AX + ( B * CY ) ) / ( D - E ) )

\item Infix Expression: ( ( A + B ) * ( C + E ) )

\item Infix Expression: ( AX * ( BX * ( ( ( CY + AY ) + BY ) * CX ) ) )

\item Infix Expression: ( ( H * ( ( ( ( A + ( ( B + C ) * D ) ) * F ) * G ) * E ) ) + J )
\end{enumerate}

\newpage
\section{Solutions to Section~\ref{sec:convert}}

\begin{enumerate}
\item
Infix Expression: ( AX + ( B * C ) )
\\
Postfix Expression: AX B C * +
\\
Prefix Expression: + AX * B C

\item
Infix Expression: ( ( AX + ( B * CY ) ) / ( D - E ) )
\\
Postfix Expression: AX B CY * + D E - /
\\
Prefix Expression: / + AX * B CY - D E

\item
Infix Expression: ( ( A + B ) * ( C + E ) )
\\
Postfix Expression: A B + C E + *
\\
Prefix Expression: * + A B + C E

\item
Infix Expression: ( AX * ( BX * ( ( ( CY + AY ) + BY ) * CX ) ) )
\\
Postfix Expression: AX BX CY AY + BY + CX * * *
\\
Prefix Expression: * AX * BX * + + CY AY BY CX

\item
Infix Expression: ( ( H * ( ( ( ( A + ( ( B + C ) * D ) ) * F ) * G ) * E ) ) + J )
\\
Postfix Expression: H A B C + D * + F * G * E * * J +
\\
Prefix Expression: + * H * * * + A * + B C D F G E J
\end{enumerate}

%\begin{thebibliography}{9}
%    \bibitem{deep_learning}
%    Lecun, Y., Bengio, Y., and Hinton, G. (2015).
%    Deep learning.
%    Nature, 521(7553), 436-444.
%\end{thebibliography}
\end{document}