\documentclass{article}
\usepackage[utf8]{inputenc}
\usepackage{graphicx}
\graphicspath{ {./} }
\usepackage{amsmath}

\title{ACSL Final Test}
\author{Alexander Sun}
\date{April 2019}

\begin{document}
\maketitle

\newpage

\section{Computer Number Systems}
\begin{enumerate}
    \item{Problem 1 (3) Let $N$ be the number of positive integers that are less than or equal to $2003$ and whose base-$2$ representation has more $1$'s than $0$'s. Find the remainder when $N$ is divided by $1000$}
    
    \item{Problem 2 (2) Convert to base 4, 8, 16 and 32}
    
    010111010100010101110101010001010111010100010001011110010010010101010101010010100
    
    101001110100010
\end{enumerate}
\section{Recursive Functions}
\begin{enumerate}

    \item{Problem 1 (3) Find A(4,3):}
    
    Ackerman's function is defined as:

\[ A(x,y)  \left\{
\begin{array}{ll}
      y + 1 & if x = 0 \\
      A(x - 1, 1) & if x \neq 0 and y = 0 \\
      A(x - 1,A(x, y - 1)) & if x \neq 0 and  y \neq 0 \\
\end{array} 
\right. \]

    \item{Problem 2 (1) Find hanoi(26)}
    \[ hanoi(x)  \left\{
\begin{array}{ll}
      1 & if x = 1 \\
      2 * hanoi(n-1) + 1 & if x > 1 \\
      
\end{array} 
\right. \]
\end{enumerate}
\section{Program Simulation}
\begin{enumerate}
    \item {Problem 1 (1) Find C[4]}
    \begin{description}
    \item A(0) = 12: A(1) = 41: A(2) = 52
\item A(3) = 57: A(4) = 77: A(5) = -100
\item B(0) = 17: B(1) = 34: B(20 = 81
\item J = 0: K = 0: N = 0
\item while A(J) $\textgreater$ 0
\item while B(K) $\leq$ A(J)
\item C(N) = B(K)
\item N = N + 1
\item k = k + 1
\item end while
\item C(N) = A(J): N = N + 1: J = J + 1
\item end while
C(N) = B(K)
    \end{description}
    \item{Problem 2 (1) Find the final value of NUM}
    \begin{description}
    \item A = “BANANAS”
    \item NUM = 0: T = “”
    \item for J = len(A) - 1 to 0 step –1
    \item T = T + A[j]
    \item next 
    \item for J = 0 to len(A) - 1
    \item if A[J] == T[J] then NUM = NUM + 1
 \item next
    \end{description}
\end{enumerate}




\section{Prefix/Postfix/Infix (Satan  Math)}
\begin{enumerate}
    \item{Problem 1 (1) Evaluate}
    
    Define  @ a b c $= Max(a,b) / Min(b,c)$
    
    Define $!\:a = a!$
    $$@\:(@\:4\:19\:3)\:(\wedge\:+\:*\:3\:4\:/\:8\:2\:–\:7\:5)\:(!\:@\:7\:9\:12)$$
    \bigskip
    
     \item{Problem 2 (1) Convert to Prefix and Infix}
    $$\frac{a * b^2}{c+1} - \frac{a + b}{a^2 * b}$$
    \bigskip

\end{enumerate}

\section{Bit String Flicking}
\begin{enumerate}
    \item{Problem 1 (3) What combination of operations does X represent?}
    
    1111110111000 XNOR (111010111111 NOR (0001010001001 XOR NOT LCIRC-4 ((RSHIFT-3 1111111111111) OR 00000000000000)))
    = X(1110111011111)
\end{enumerate}
\section{LISP}
\begin{enumerate}
    \item{Problem 1 (1) Solve:}
    Given the function definitions for HY and FY as follows:
    
    (DEF HY(PARMS)(REVERSE(CDR PARMS)))
    
    (DEF FY(PARMS)(CAR(HY(CDR PARMS))))
    
    What is the value of the following?
    
    (FY'(DO RE(MI FA)SO))
    \item{Problem 2 (1)}
    Evaluate the following expression:
    
    (EXP ( MULT 2(SUB 5(DIV(ADD 5 3 4) 2 )) 3) 3)
\end{enumerate}
\section{Boolean Algebra}
\begin{enumerate}
    \item{Problem 1 (3)} 
    Prove:
    $$\overline{x} + \overline{y} + xy\overline{z} = \overline{x} + \overline{y} + \overline{z}$$
    \bigskip
    \bigskip
    
    \item{Problem 2 (2)}
    Simplify: 
    $$(p + q)(\overline{pq})$$
    \bigskip
    
    \item{Problem 3 (1)}Assume a 50/50 probability for the value of each variable. What is the probability that x = 0?:
    $$X = (A+B)(A+\overline{C})(A+D)(A+\overline{E})(A+F)(A+\overline{G})....$$
   
    
    \bigskip
    \bigskip
    \bigskip
    
    
\end{enumerate}
\section{Data Structures}
\begin{enumerate}
    \item{Problem 1 (2)} Put the following phrase into a Min-Heap, What are the letters that fill the bottom row and of what frequency?
    
    wecanonlyseeashortdistanceaheadbutwecanseeplentytherethatneedstobedonebyalanturing
    
    \bigskip
    \item{Problem 2 (2)} What would be its external path and internal path length?
    \bigskip
    \item{Problem 3 (2)} What search algorithms are derived from a stack and a queue, and what are they used for and why within each algorithm?
    \bigskip
    
    
\end{enumerate}


\section{FSA/Regular Expressions}
\begin{enumerate}
    \item{Problem 1 (2) Simplify the following regex expression(no plus symbol)}
    
    $$a*(ba*)* (d*c*)* (e* U f)*(gh)*g(ij U ik)$$
    
    \item{Problem 2 (1) Does the following string comply with the regex above?}
    
    $$aabadccddceeeffggghghij$$

\end{enumerate}
\section{Graph Theory}
\begin{enumerate}
    \item {Problem 1 (1) Multiply the following matrices}
    $$
    \begin{bmatrix} 
    -1 & 2 \\
    5 & 4 \\
    -4 & -3 \\
    -1 & 0
    \end{bmatrix}
    \begin{bmatrix} 
    7 & 8 & 9 \\
    4 & -3 & 2 
    \end{bmatrix}
    $$
    \item {Problem 2 (1) Multiply the following matrices}
     $$
    \begin{bmatrix} 
    1 & 0 & 1 & 0 & 0 & 0 \\
    0 & 1 & 1 & 1 & 0 & 0 \\
    0 & 0 & 0 & 0 & 1 & 1\\
    1 & 0 & 1 & 0 & 0 & 0 \\ 
    1 & 0 & 1 & 0 & 0 & 0 \\
    0 & 0 & 0 & 0 & 1 & 1
    \end{bmatrix}
    \begin{bmatrix} 
    1 & 0 & 1 & 0 & 0 & 0 \\
    0 & 1 & 1 & 1 & 0 & 0 \\
    0 & 0 & 0 & 0 & 1 & 1\\
    1 & 0 & 1 & 0 & 0 & 0 \\ 
    1 & 0 & 1 & 0 & 0 & 0 \\
    0 & 0 & 0 & 0 & 1 & 1
    \end{bmatrix}
    $$
\end{enumerate}

\section{Digital Electronics}
\begin{enumerate}
    \item {Problem 1 (2)}
    
    \includegraphics[scale=.75]{DigitalElectronicsACSL}
\end{enumerate}
\section{Assembly Language}
\begin{enumerate}
    \item {Problem 1 (2) Write the following Assembly code as a piece wise function as find what is printed at the end(Ignore the messed up indentation)}

\begin{description}
\item X\quad\quad\enspace DC \quad\quad\quad\enspace 3
\item Y\quad\quad\enspace DC \quad\quad\quad\enspace 5
\item Check LOAD \quad\quad\enspace X
\item \quad\quad\quad BE \quad\quad\quad\enspace Yellow
\item  \quad\quad\quad LOAD\quad\quad\enspace Y
\item \quad\quad\quad BE \quad\quad\quad\enspace Red
\item \quad\quad\quad BU \quad\quad\quad\enspace Blue
\item Blue \space\space LOAD \quad\quad\enspace X
\item \quad\quad\quad ADD \quad\quad\enspace\enspace 1
\item \quad\quad\quad STORE \quad\enspace X
\item \quad\quad\quad LOAD\quad\quad\enspace Y
\item \quad\quad\quad SUB \quad\quad\enspace\enspace 1
\item \quad\quad\quad STORE \quad\enspace Y
\item Red \space\space LOAD\quad\quad\enspace X
\item \quad\quad\quad  SUB \quad\quad\enspace\enspace 1
\item \quad\quad\quad STORE \quad\enspace X
\item Yellow LOAD\quad\quad\enspace X
\item \quad\quad\quad  ADD  \quad\quad\enspace\enspace 1
\item \quad\quad\quad STORE \quad\enspace X
\item \quad\quad\quad PRINT \quad\enspace X
\end{description}
    \item{Problem 2 (1) Define LOC, OPCODE, LABEL, and ACC and place them in order of a command}
\end{enumerate}

\end{document}
