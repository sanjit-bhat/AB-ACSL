% Inspired by Dan Spielman's template

\documentclass[10pt]{article}
\usepackage[T1]{fontenc}
\usepackage{amssymb}
\usepackage{amsmath}
\usepackage{graphicx}

\usepackage{tikz}
\usetikzlibrary{arrows}
\usepackage{subfigure}
\usepackage{stackrel}
\usepackage{blindtext}

\oddsidemargin=0.15in
\evensidemargin=0.15in
\topmargin=-.5in
\textheight=9in
\textwidth=6.25in

\usepackage[colorlinks=true,breaklinks,pdfpagemode=none,linkcolor=blue, urlcolor = blue, citecolor=blue]{hyperref}
\usepackage{enumerate}

%\usepackage{enumitem}
%\setlist{itemsep=0mm}

%\usepackage[usenames,dvipsnames]{pstricks}
%\usepackage{epsfig}
\usepackage{amsmath,amsfonts,amssymb,bm}
%\usepackage{pst-grad} % For gradients
%\usepackage{pst-plot} % For axes

% Enviroment definitions (add your own here)

\newtheorem{theorem}{Theorem}
\newtheorem{corollary}[theorem]{Corollary}
\newtheorem{lemma}[theorem]{Lemma}
\newtheorem{observation}[theorem]{Observation}
\newtheorem{proposition}[theorem]{Proposition}
\newtheorem{definition}[theorem]{Definition}
\newtheorem{claim}[theorem]{Claim}
\newtheorem{fact}[theorem]{Fact}

\newenvironment{proof}{\noindent{\bf Proof}\hspace*{1em}}{\qed\bigskip}

% New commands (add your own here)

\newcommand{\eps}{\varepsilon}
\newcommand{\bbR}{\mathbb{R}}
\newcommand{\hv}{\hat{v}}
\newcommand{\hL}{\hat{L}}
\newcommand{\hlambda}{\hat{\lambda}}
\newcommand{\homega}{\hat{\omega}}
\newcommand{\hp}{\hat{p}}
\newcommand{\hW}{\hat{W}}
\newcommand{\cK}{\mathcal{K}}
\newcommand{\qed}{\rule{7pt}{7pt}}
\newcommand{\cF}{\mathcal{F}}

\begin{document}

    \noindent
    \begin{center}

        \hrulefill

        \vspace{5pt}

        \makebox[\textwidth]{ {\bf AB IdeaLab, Competitive Programming Team, Fall 18--Spring 19} \hfill  February 8, 2019}
        \vspace{0pt}

        {\Large \hfill  Lecture 3: FSA/Regular Expressions\hfill}
        \vspace{10pt}

        {\large \hfill  American Computer Science League, February Contest\hfill}
        \vspace{10pt}

        \makebox[\textwidth]{ {\it Lecturer: Sanjit Bhat \hfill Editor: Alexander Sun} } % primary editor is the 'lecturer'

        \vspace{-3pt}
        \hrulefill
    \end{center}

\section{Fun Facts}
\begin{itemize}
\item Developed in 1951 by mathematician Stephen Cole Kleene.
\item Ken Thompson (one of the guys who developed UNIX) used regular expressions on an early Unix editor.
This eventually lead to its use in the famous UNIX tool grep.
\item Applications include string searching algorithms,  input verification, and search engines.
\item You can even use it inside your programming editor to find where you've put stuff.
\end{itemize}

\section{Background}
\paragraph{What are regular expressions?}
According to Wikipedia, regular expressions (regex) are ``a sequence of characters that define a search pattern''.
In other words, a regex defines a set of possible strings in a concise manner for
some later purpose.
For example, reali[sz]e defines the set \{realize, realise\} of possible strings.
These set can be later used for cross-referencing American English spellings with British
English spellings.

\paragraph{How are they interpreted by the computer?}
In a regex, there are two types of chars: literals and metacharacters.
Literals define regular characters, while metacharacters indicate
more nuanced behaviors.
After creating a regex, a regex processor transforms the characters into an internal
representation that computers can process.
This representation in known as a Finite State Automata (FSA).
FSA's are a broad mathematical model consisting of the following:
\begin{enumerate}
\item A finite number of states, of which exactly one is active at any given time
\item Transition rules to change the active state
\item An initial state
\item One or more final states
\end{enumerate}
We can draw an FSA by representing each state as a circle, the final state
as a double circle, the start state as the only state with an incoming arrow,
and the transition rules as labeled-edges connecting the states.

\paragraph{All the regex syntax you need to know.}
As mentioned above, regex includes metacharacters that define more complex types
of string matching.
The following is a list of all the regex metacharacters you need to know:
\begin{enumerate}
\item \textbf{|, or, $\cup$}
These are booleans that tell the processor to take the set union of the left
and right regexes.
\item \textbf{$\lambda$}
The null or empty string.
\item \textbf{Quantification}
Defining the number of something allowed to occur.
    \begin{enumerate}
    \item \textbf{?}
    Zero or one.
    E.g., colou?r = \{color, colour\}.
    \item \textbf{*}
    Zero or more.
    \item \textbf{+}
    One or more.
    \end{enumerate}
\item \textbf{.}
Wildcard, or any character.
Combine . and * for a.*b, which means any string with a and b as the left and right
characters with anything inbetween.
\item \textbf{[\ldots]}
Set of possible character matches.
Think reali[sz]e example above.
This can get slightly more complex by using hyphens to define ranges of possible characters.
E.g., [a-z] means every \textit{lowercase} char from a to z;
[abcx-z] means a, b, c, and x, y, z; and [a-cx-z] means a, b, c and x, y, z.
\item \textbf{[\string^\ldots]}
Set of characters not contained withing the brackets.
E.g., [\string^a-z] matches any characcter that is not a lowercase letter from a to z.
\item \textbf{()}
Just like in math, parentheses imply grouping.
E.g., if we wanted the set \{gray, grey\}, gra|ey would give us \{gra, ey\}.
Instead, using parentheses we can get gr(a|e)y, which gives us the correct regex.
A more complex example is H(\"{a}|ae?)ndel, which matches \{Handel, H\"{a}ndel, Haendel\}.
\end{enumerate}

\paragraph{Practicing the syntax via identity proofs.}
To make sure you understand the syntax, see if you can prove the following identities:
\begin{enumerate}
\item (a*)* = a*
\item aa* = a*a
\item aa* $\cup \lambda$ = a*
\item a(b $\cup$ c) = ab $\cup$ ac
\item a(ba)* = (ab)*a
\item (a $\cup$ b)* = (a* $\cup$ b*)*
\item (a $\cup$ b)* =(a*b*)*
\item (a $\cup$ b)* = a*(ba*)*
\end{enumerate}

\paragraph{Practicing regex syntax.}
If you would like to practice regex and have your code actually
matched against strings, I recommend \href{https://regexr.com/}{this} website.

\section{Exercises}
\subsection{Translate an FSA to a Regular Expression}
\label{par:translate}
\begin{enumerate}
\item \href{http://www.categories.acsl.org/wiki/index.php?title=FSAs_and_Regular_Expressions}{ACSL Wiki page}.
\item \href{http://www.apcomputerscience.com/cst/topic_descriptions/regularExpressionsAndFSA.pdf}{Random website with diagrams.}
\end{enumerate}

\subsection{Simplify a Regular Expression}
\label{par:simplify}

\subsection{Determine which Regular Expressions or FSAs are equivalent}
\label{par:equivalent}

\subsection{Determine which strings are accepted by either an FSA or a Regular Expression}
\label{par:accepted}
\begin{enumerate}
\item Which of the following strings are accepted by the following Regular Expression     ``00*1*1U11*0*0''?
    \begin{enumerate}
    \item 0000001111111
    \item 1010101010
    \item 1111111
    \item 0110
    \item 10
    \end{enumerate}
\item Which of the following strings match the regular expression
pattern ``[A-D]*[a-d]*[0-9]''?
    \begin{enumerate}
    \item ABCD8
    \item abcd5
    \item ABcd9
    \item AbCd7
    \item X
    \item abCD7
    \item DCCBBBaaaa5
    \end{enumerate}
\item Which of the following strings match the regular expression
pattern ``Hi?g+h+[\string^a-ceiou]''?
    \begin{enumerate}
    \item Highb
    \item HiiighS
    \item HigghhhC
    \item Hih
    \item Hghe
    \item Highd
    \item HgggggghX
    \end{enumerate}
\end{enumerate}

%\newpage
\section{Solutions}

\subsection{Answers for Section~\ref{par:accepted}}
\begin{enumerate}
\item 0000001111111 and 10
\item ABCD8, abcd5, ABcd9, and DCCBBBaaaa5
\item HigghhhC, Highd, and HgggggghX
\end{enumerate}

%\begin{thebibliography}{9}
%    \bibitem{deep_learning}
%    Lecun, Y., Bengio, Y., and Hinton, G. (2015).
%    Deep learning.
%    Nature, 521(7553), 436-444.
%\end{thebibliography}
\end{document}