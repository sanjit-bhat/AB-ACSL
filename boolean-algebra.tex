% Inspired by Dan Spielman's template

\documentclass[10pt]{article}
\usepackage[T1]{fontenc}
\usepackage{graphicx}

\usepackage{tikz}
\usetikzlibrary{arrows}
\usepackage{subfigure}
\usepackage{stackrel}
\usepackage{blindtext}

\oddsidemargin=0.15in
\evensidemargin=0.15in
\topmargin=-.5in
\textheight=9in
\textwidth=6.25in

\usepackage[colorlinks=true,breaklinks,pdfpagemode=none,linkcolor=blue, urlcolor = blue, citecolor=blue]{hyperref}
\usepackage{enumerate}

%\usepackage{enumitem}
%\setlist{itemsep=0mm}

%\usepackage[usenames,dvipsnames]{pstricks}
%\usepackage{epsfig}
\usepackage{amsmath,amsfonts,amssymb,bm}
%\usepackage{pst-grad} % For gradients
%\usepackage{pst-plot} % For axes

% Enviroment definitions (add your own here)

\newtheorem{theorem}{Theorem}
\newtheorem{corollary}[theorem]{Corollary}
\newtheorem{lemma}[theorem]{Lemma}
\newtheorem{observation}[theorem]{Observation}
\newtheorem{proposition}[theorem]{Proposition}
\newtheorem{definition}[theorem]{Definition}
\newtheorem{claim}[theorem]{Claim}
\newtheorem{fact}[theorem]{Fact}

\newenvironment{proof}{\noindent{\bf Proof}\hspace*{1em}}{\qed\bigskip}

% New commands (add your own here)

\newcommand{\eps}{\varepsilon}
\newcommand{\bbR}{\mathbb{R}}
\newcommand{\hv}{\hat{v}}
\newcommand{\hL}{\hat{L}}
\newcommand{\hlambda}{\hat{\lambda}}
\newcommand{\homega}{\hat{\omega}}
\newcommand{\hp}{\hat{p}}
\newcommand{\hW}{\hat{W}}
\newcommand{\cK}{\mathcal{K}}
\newcommand{\qed}{\rule{7pt}{7pt}}
\newcommand{\cF}{\mathcal{F}}

\begin{document}

    \noindent
    \begin{center}

        \hrulefill

        \vspace{5pt}

        \makebox[\textwidth]{ {\bf AB IdeaLab, Competitive Programming Team, Fall 18--Spring 19} \hfill  February 1, 2019}
        \vspace{0pt}

        {\Large \hfill  Lecture 1: Boolean Algebra\hfill}
        \vspace{10pt}

        {\large \hfill  American Computer Science League, February Contest\hfill}
        \vspace{10pt}

        \makebox[\textwidth]{ {\it Primary Editor: Alexander Sun \hfill Secondary Editor: Sanjit Bhat} }

        \vspace{-3pt}
        \hrulefill
    \end{center}


\section{Introduction}
Boolean algebra is the branch of algebra in which the variables store truth value. All variables are true(1) or false(0).
There are 3 main operations that create the base for boolean algebra: $\textbf{AND(conjunction)}$, $\textbf{OR(disjunction)}$, and $\textbf{NOT(negation)}$.

\bigskip
\noindent
\section{Basic Operations}
\textbf{AND}, denoted x $\land$ y or x AND y or x $\cdot$ y, satisfies x $\land$ y = 1 if x = y = 1 , else x $\land$ y = 0

\noindent
\textbf{OR}, denoted x $\lor$ y or x OR y or x + y, satisfies x $\lor$ y = 0 if x = y = 0 , else x $\lor$ y = 1

\noindent
\textbf{NOT}, denoted $\neg$x or NOT x or $\sim$x or $\bar{x}$, satisfies $\neg$x = 0 if x = 1 and $\neg$x = 1 if x = 0, reverses truth values of operation

\begin{center}
\begin{tabular}{ |c|c|c|c| }
 \hline
 x & y & x$\land$y & x$\lor$y \\
 \hline
  0 & 0 & 0 & 0 \\
 \hline
 1 & 0 & 0 & 1 \\
 \hline
 0 & 1 & 0 & 1 \\
 \hline
 1 & 1 & 1 & 1 \\
 \hline
\end{tabular}
\end{center}

\begin{center}
\begin{tabular}{ |c|c| }
 \hline
 x & $\neg$x \\
 \hline
  0 & 1  \\
 \hline
 1 & 0 \\
 \hline
\end{tabular}
\end{center}

\section{Secondary Operations}

\textbf{Material Implication}, denoted  x $\rightarrow$ y = $\neg$x$\lor$y, if x = 1, then x $\rightarrow$ y = y, if x = 0, then x $\rightarrow$ y = 1

\noindent
\textbf{Exclusive Or}, denoted x $\oplus$ y or x XOR y, x $\oplus$ y = 1 if x = 1 $\&$ y = 0 or x = 0 $\&$ y = 1, else x $\oplus$ y = 0, true when values are different

\noindent
\textbf{Equivalence}, denoted x $\equiv$ y, x $\equiv$ y = 1 if x =1 $\&$ y =1, or if x = 0 $\&$ y = 0, complement of XOR, true when values are the same
\noindent
\textbf{Dual}, the dual is found by replacing all OR's with AND's and all AND's with OR's, and all 1's with 0's and all 0's with 1's

\noindent
\textbf{Complement}, found by negating each individual value and replaving all OR's with AND's and all AND's with OR's and all 1's with 0's and all 0's with 1's
\bigskip

\noindent
Memorizing boolean algebra laws is extremely beneficial to being able to solve problems quickly and efficiently. Here is a link to page with almost every law. Most are derivable, but should still be memorized.  http://www.uiltexas.org/files/academics/UILCS-BooleanIdentities.pdf

Demorgan's Rule is crucial to simplifying boolean algebra problems. It states: $$ \bar{A} + \bar{B} = \overline{AB}$$ or $$ \bar{A} * \bar{B} = \overline{A+B}$$

\bigskip

\noindent
With these basic rules memorized, all boolean algebra problems should be simple to work through

\section{Exercises}
\begin{enumerate}
\item Simplify completely:(ACSL 2001-2002)
$( A + B )\oplus A B$

\item Simplify the following expression: F = BC + $\overline{BC}$ + BA

\item Simplify the Boolean expression (A+B+C)$\overline{(D+E)}$ + (A+B+C)(D+E):
\end{enumerate}

For more practice resources on boolean algebra, read through
\href{http://www.categories.acsl.org/wiki/index.php?title=Boolean_Algebra}{the ACSL Wiki page}.
%\begin{thebibliography}{9}
%    \bibitem{deep_learning}
%    Lecun, Y., Bengio, Y., and Hinton, G. (2015).
%    Deep learning.
%    Nature, 521(7553), 436-444.
%\end{thebibliography}
\end{document}










